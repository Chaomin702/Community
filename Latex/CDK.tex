\documentclass[UTF8]{ctexart}
\usepackage[paper=a4paper,dvips,top=2.5cm,left=2.8cm,right=2.8cm,foot=1cm,bottom=3.2cm]{geometry}
\usepackage{fancyhdr}
\usepackage{indentfirst}
\usepackage{enumerate}
\usepackage{clrscode}
\usepackage{listings}
\usepackage{amsmath}
\usepackage{amstext}
\lstset{language=Matlab}%代码语言使用的是matlab
\lstset{breaklines}%自动将长的代码行换行排版
\lstset{extendedchars=false}%解决代码跨页时,章节标题,页眉等汉字不显示的问题
\setlength{\parskip}{0.5\baselineskip}
\usepackage{graphicx}
\usepackage[colorlinks,linkcolor=blue,urlcolor = blue]{hyperref}
\DeclareGraphicsExtensions{.eps,.ps,.jpg,.bmp}
\pagestyle{plain}
\begin{document}
\par 尊敬的吴老师,您好
\newline
\par
我最近看了几篇关于网络信息传播模型的论文,重点阅读了这篇论文:“Learning Social Network Embeddings for Predicting Information Diffusion”,这篇论文提出的CDK(Content diffusion Kernel)模型简洁明了,我想先从这篇论文开始。下面是我对这么论文的一些理解。

\par CDK模型的创新之处在于大胆地抛弃了网络的拓扑结构,转而用热传导的现象来对信息传播建模。具体的:我们先将网络中的节点映射到N维欧氏空间中,假设某时刻节点A转发了某条消息,那么我们想象在节点A处亮起一簇火苗,随着时间的流逝,节点A附近的节点温度不断升高,一旦某节点(假设为B)的温度达到某个阈值,那么B被激活。在给定的时间内,按时间先后依次激活的节点序列则构成一条转发链。
\par CDK模型的训练过程就是根据训练样本,不断调整节点的坐标。
当然,我们并不需要具体的去模拟这一物理过程来训练模型,训练过程可以简洁地转化为一个排序过程。

相较于传统的IC模型,CDK的计算速度很快,大概分别是minutes与days的量级。
在没有网络拓扑结构的情况下,CDK的预测精度能与IC模型旗鼓相当,甚至更好。

作者开源了论文代码,基于Torch7框架实现。我从头到尾读了一遍。

\par 老师您觉得这篇论文可以做下去吗?
\par \rightline{学生王超民,2016年9月25日}

\end{document}
