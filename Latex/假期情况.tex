\documentclass[UTF8]{ctexart}
\usepackage[paper=a4paper,dvips,top=2.5cm,left=2.8cm,right=2.8cm,foot=1cm,bottom=3.2cm]{geometry}
\usepackage{enumerate}
\usepackage[colorlinks,linkcolor=blue,urlcolor = blue]{hyperref}
\begin{document}
\par 尊敬的吴老师,您好
\newline
\par 暑假期间我回家考驾照,同时也把搁置了一段时间的“微博转发预测”重新过了一遍。
\par 之前在做“转发链长度预测”仿真时,遇到的问题是计算得到的MAPE值小十倍,转发链长度波峰与真实数据差别很大。
造成这种结果的原因,我分析如下:
\begin{enumerate}[\indent 1)]
\item 公式中的参数设置不合理
\item 预测过程理解不到位
\item 程序有bug
\end{enumerate}
\par 对于原因$1$:我的思路是穷举。参数一共有6个,分别设定范围和步长,我写了一个python脚本来统计所有参数组合的输出(在一台廉价云服务器上跑了20天),取最优,但结果仍旧与论文参考值相差一个数量级。
\par 对于原因$2$:论文中有一个公式跟我的推导结果不一致;在作者提供的数据中,有一些未说明的符号;以及最让我费解的一点:对一个“看起来”非凸的函数做梯度下降,而且下降后期出现的锯齿现象导致了庞大的计算量...我有向作者发邮件请教过这些问题(见questions.pdf),然而石沉大海,杳无音信..
\par 对于原因$3$:我的方法是用手工计算结果与程序结果比对,至少前几条消息的结果是一致的。
\newline
\par 我有些“不知所措”了,老师您能不能给个主意?
\newline
\par \rightline{学生王超民,2016年8月26日}
\end{document}