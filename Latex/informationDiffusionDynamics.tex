\documentclass[UTF8]{ctexart}
\usepackage[paper=a4paper,dvips,top=2.5cm,left=2.8cm,right=2.8cm,foot=1cm,bottom=3.2cm]{geometry}
\usepackage{fancyhdr}
\usepackage{indentfirst}
\usepackage{enumerate}
\usepackage{clrscode}
\usepackage{listings}
\usepackage{amsmath}
\usepackage{amstext}
\lstset{language=Matlab}%代码语言使用的是matlab
\lstset{breaklines}%自动将长的代码行换行排版
\lstset{extendedchars=false}%解决代码跨页时,章节标题,页眉等汉字不显示的问题
\setlength{\parskip}{0.5\baselineskip}
\usepackage{graphicx}
\usepackage[colorlinks,linkcolor=blue,urlcolor = blue]{hyperref}
\DeclareGraphicsExtensions{.eps,.ps,.jpg,.bmp}
\pagestyle{plain}
\begin{document}
\par 尊敬的吴老师,您好
\newline
\par
这两天我继续阅读了有关信息传播动力学的几篇论文:
\begin{enumerate}[\indent 1)]
\item
“Fast Information Cascade Prediction Through Spatiotemporal Decompositions”,作者提出一种方法:将转发链信息综合为一个矩阵,将该矩阵进行SVD分解得到用户特征(影响力和敏感度),进而做一定时间内转发链长度的预测。
\item
“PREDICTING INFORMATION DIFFUSION VIA MATRIX FACTORIZATION BASED MODEL”中,直接从观察到的时间序列对信息传播建模。作者假设不同话题的信息具有不同的传播特征,从而对不同的话题建立传播网络,然后利用矩阵分解对已有的信息进行建模,进而预测未来的信息传播过程。
\item
“MODELING INFORMATION DIFFUSION DYNAMICS OVER SOCIAL NETWORKS”则考虑到了用户转发信息时的决策行为,提出了基于evolutionary game theoretic的信息传播动力学模型。
\end{enumerate}
\par 老师您能否给个建议,我应该哪篇论文开始深入?
\newline
\newline
\par \rightline{学生王超民,2016年9月27日}

\end{document}
