\documentclass[UTF8]{ctexart}
\usepackage[paper=a4paper,dvips,top=2.5cm,left=2.8cm,right=2.8cm,foot=1cm,bottom=3.2cm]{geometry}
\usepackage{enumerate}
\usepackage[colorlinks,linkcolor=blue,urlcolor = blue]{hyperref}
\begin{document}
\par 尊敬的吴老师,您好!
\newline
\par 我想把最近这半个月来的学习情况跟您汇报一下。
\par 对于“经过指定节点集的最短路径”问题,我们主要的工作围绕着这篇文献“Protected shortest path visiting specified nodes”展开。目前的进度如下:
\begin{itemize}
\item 读了《算法导论》关于图算法的章节,并实现了其中部分算法。
\item 大致弄懂了“Protected shortest path visiting specified nodes”的第三节内容,并对该部分的算法做了实现
\url{https://github.com/Chaomin702/shortestPath.git}。
\end{itemize}
\par 目前的测试结果是:对于节点规模为200,特定节点规模为20的测例,大概能在3秒内得出结果,但结果中存在环路,且环路发生在非特定节点中。可以说我们现在得到的结果,是一个\textbf{相对较优的特定节点集排列},但在依次连接特点节点对时,存在了环路。
\par 论文第四节的ASK、BSK算法可以确保路径无环,但方法晦涩难懂,我们尚且只弄清楚了ASK的初始化步骤。
\newline
\par 过程中的收获
\begin{itemize}
\item 刚看那篇论文时,完全不知所云。在与党妮、李晓杰同学数次的讨论过程中,逐渐弄清楚了论文内容。正如您所说,跟他人讨论交流中,很容易打开思路。
\item 先后将算法的运行时间从20s优化到了10s,到现在的3s,并且仍存在优化的空间(500ms内)。在这个过程中,编码的能力得到了提升。
\item 写文档的过程中,思路得到了进一步的梳理。
\end{itemize}

\end{document}